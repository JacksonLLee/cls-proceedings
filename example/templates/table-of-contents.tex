\documentclass[12pt]{article}
\usepackage{tipa}
\usepackage{times}
\usepackage[utf8]{inputenc}
\usepackage[T1]{fontenc}
\usepackage[left=1.5in,top=1.25in,right=1.5in,bottom=1.25in]{geometry}

\hyphenpenalty=10000

% The command "TocEntry" defined here controls how each entry in the table of
% contents looks. Changes to this command are possible, as long as:
% 1) The name of this command (i.e., "TocEntry") is not changed.
% 2) This command takes exactly three arguments:
%       1st argument is the author(s).
%       2nd argument is the paper title.
%       3rd argument is the starting page number of the paper.

\newcommand{\TocEntry}[3]{

\noindent
\begin{tabular}{p{4.9in} p{.3in}}
\textsc{#1} &  \mbox{} \hfill #3\\
\begin{tabular}{p{.3in} p{4.5in}} & \textit{#2} \end{tabular}\\
\end{tabular}\vspace{1em}}

\begin{document}

\pagenumbering{gobble} % disable all page numbers

\begin{center}
{\fontfamily{phv}\selectfont \bfseries Table of Contents}
\end{center}

\vspace{1em}

% A TOC entry looks like this:
% \TocEntry{AUTHORS}{PAPERTITLE}{STARTPAGE}

% An example of a TOC entry:
% \TocEntry{John Alderete, Paul Tupper, and Stefan A. Firsch}{Phonotactic learning without \emph{a priori} constraints: Arabic root cooccurrence restrictions revisited}{1}

% don't change the following line in this template!
XXInsertTocEntriesXX

\end{document}

